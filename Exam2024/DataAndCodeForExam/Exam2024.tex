\documentclass[11pt]{article}

\usepackage[latin1]{inputenc}
\usepackage[danish]{babel}        % Use English headings, date format.
\usepackage{a4wide}               % A4 (DIN format).
\usepackage[hidelinks]{hyperref}  % Enable direct links in PDF (e.g. for data sets)
\usepackage{mathabx}

\textheight=1.04\textheight
\textwidth=1.10\textwidth
\hoffset=-0.05\textwidth
\leftmargin=-0.18\textwidth
\headsep=0.0pt
\headheight=0.0pt

\vfuzz2pt   % Don't report over-full v-boxes if over-edge is small
\hfuzz10pt  % Don't report over-full h-boxes if over-edge is smallish

\newcommand{\half}{\mbox{$\frac{1}{2}$}}


\begin{document}
%\pagestyle{empty}

%----------------------------------------------------------------------------
\noindent
University of Copenhagen \hfill
Niels Bohr Institute, \today \par
\vspace{-2ex}
\noindent
\hrulefill

\vspace{1ex}
\begin{center}
{\bf {\Huge Applied Statistics}}\\
\vspace{1ex}
{\large Exam in Applied Statistics 2024/25}
\end{center}

%----------------------------------------------------------------------------
\vspace{0ex}
\noindent
This take-home exam was distributed Thursday the 16th of January 2025 at 08:00. A solution in PDF format must
be submitted at \texttt{\bf www.eksamen.ku.dk by 20:00 Friday the 17th}, along with all code used to work out
your solutions (as appendix). Links to data files can also be found on the course webpage and github. Working
in groups or discussing the problems with others is {\bf NOT} allowed.

\begin{center}
  Thank you for all your hard work, Malthe, Beatrice, Rashmi, Marcela, Mathias, \& Troels.
\end{center}


%----------------------------------------------------------------------------

\noindent
\hrulefill\\
\emph{The knowledge of certain principles easily compensates the lack of knowledge of certain facts.}\\
  \phantom{foobar} \hfill [Claude Adrien Helv�tius, 1759]\\[-2ex]

%----------------------------------------------------------------------------
\vspace{-2ex}
\noindent
\hrulefill

\vspace{4ex}
\noindent
{\bf I -- Distributions and probabilities:}
\vspace*{-1ex}
\begin{description}
  \item[1.1] (6 points)
    Every day, you roll a normal die, and if you get a six, you do roll the die 100 times
    and do as many pushups as you get sixes. Otherwise you don't do any pushups.
  \vspace*{-1ex}
  \begin{itemize}
    \item What is the distribution of days between doing pushups?
%    \item What is the distribution of number of pushups on days with pushups?
    \item What is the mean, median, and standard deviation of number of pushups in 30 days?
  \end{itemize}
%
  \item[1.2] (4 points)
    The Djoser pyramide in Egypt is North-South aligned to 3 degrees.
  \vspace*{-1ex}
  \begin{itemize}
    \item Estimate the probability that the pyramid is North-South aligned by coincidence.
  \end{itemize}
\end{description}



%----------------------------------------------------------------------------

\vspace{1ex}
\noindent
{\bf II -- Error propagation:}
\vspace*{-1ex}
\begin{description}
\item[2.1] (5 points)
%  Data from: https://onlinelibrary.wiley.com/doi/epdf/10.1111/j.1945-5100.2000.tb01518.x
  Water on Earth ($\Earth$) has a Deuterium to Hydrogen ratio of $r_{\Earth} = (149 \pm 3) \times 10^{-6}$.
  The hydrogen of the proto-solar system ($\Sun$) has a ratio of $r_{\Sun} = (25 \pm 5) \times 10^{-6}$, while that of comets ($C$)
  have been measured to be $r_C = (309 \pm 20) \times 10^{-6}$.
  \vspace*{-1ex}
  \begin{itemize}
  \item From these numbers, what fraction of water on Earth do you estimate come from the original
    proto-solar system, and what fraction do you attribute to comets?
  \end{itemize}
%
\item[2.2] (8 points)
  You run a detector for a time interval of $\Delta t = 98.4$s, during which the detector yields $N=1971$ counts.
  The time interval uncertainty is $\sigma_{\Delta t} = 3.7$s, indepedent of $\Delta t$.
  \vspace*{-1ex}
  \begin{itemize}
    \item What is the rate $r = N/t$ and its uncertainty?
    \item How long should you measure to get a relative uncertainty on the rate $r$ below 2.5\%?
  \end{itemize}
%
\item[2.3] (14 points)  
  The file \href{http://www.nbi.dk/~petersen/data\_PylonPositions.csv}{\bf www.nbi.dk/$\sim$petersen/data\_PylonPositions.csv}
  contains position measurements (both with and without uncertainties) of four pylons for a bridge.
  \vspace*{-1ex}
  \begin{itemize}
    \item Using measurements without uncertainty, determine the four pylon positions.
    \item Using measurements with uncertainty, determine the four pylon positions.
    \item Combine the two measurements. Do they match each other?
    \item Test if the four measured pylon positions are equidistant.
    \item The pylon distance should be 40.83m with a tolerance (i.e.\ maximally allowed deviation) of 1.05m.
      Do the pylon positions live up to this requirement?   % at the 95\% confidence level?
  \end{itemize}
\end{description}


%----------------------------------------------------------------------------

\vspace{1ex}
\noindent
{\bf III -- Simulation / Monte Carlo:}
\vspace*{-1ex}
\begin{description}
\item[3.1] (8 points)
  Circles $A$ and $B$ are centered at (0,0) and (3,7) and have radii of 6 and 4, respectively.
  \vspace*{-1ex}
  \begin{itemize}
    \item What fraction of $A$ overlaps with $B$? And conversely, what fraction of $B$ overlaps with $A$?
    \item If the circles were 4D ``hyperballs'' centered at (0,0,0,0) and (3,7,-1,2), respectively, and with the same radii,
      what would the answers then be?
  \end{itemize}
%
\item[3.2] (15 points)
  You want to simulate the radial material distribution $m(r)$ from a uniform explosion.\\
  \vspace*{-4ex}
  \begin{itemize}
    \item Generate 50000 $x$, $y$, and $z$ values in the range $[-1,1]$ and plot the spherical
      coordinate $r$.
    \item Selecting only points with $z > 0$ and $r < 1$, what distributions in $\theta$ and $\phi$  do you obtain?
    \item How would you produce random velocities $v$ according to $f(v) = (v/v_0)^2 \exp(-v/v_0)$? 
    \item Given $v_0 = 100~$m/s and that the radial distance of material $r$ as a function of velocity is
      $r(v) = \sin(\theta) v^2/g$ ($g = 9.82 \mbox{m/s}^2$), simulate 10000 values of $\theta$ and $v$.
      Combine these to obtain values of $r$, and plot the resulting distribution $m(r)$.
  \end{itemize}
\end{description}


%----------------------------------------------------------------------------

\vspace{1ex}
\noindent
{\bf IV -- Statistical tests:}
\vspace*{-1ex}
\begin{description}
\item[4.1] (12 points)
  You get a permanently closed box with 20 normal (i.e.\ six-sided) dices in. One of the dice is potentially fake,
  with all the sides having the same (unknown) value. You can shake the box and see the resulting 20 dice roll
  as many times as you like.
  \vspace*{-1ex}
  \begin{itemize}
    \item Simulate 100 box rolls and plot the die frequencies, both with and without a fake die in.
    \item For both of your simulated datasets, test if there is a fake die or not.
    \item How many rolls would you require before you would argue, that you could tell the difference?
  \end{itemize}
\end{description}


%----------------------------------------------------------------------------

\vspace{1ex}
\noindent
{\bf V -- Fitting data:}
\vspace*{-1ex}
\begin{description}
\item[5.1] (14 points)
  The file \href{http://www.nbi.dk/~petersen/data\_InconstantBackground.csv}{\bf www.nbi.dk/$\sim$petersen/data\_InconstantBackground.csv}
  contains molecular interspacing measurements $d$ (in nm) from a scattering experiment.
  \vspace*{-1ex}
  \begin{itemize}
    \item Plot the data and test to what extend the background in the range [8,10] is uniform.
    \item Fit the three Gaussian peaks at around $d$ = 0.9, 3.4, and 5.9 nm, including local background.
    \item Test if the peaks have the same intensity, i.e.\ number of measurements in them.
    \item Try to fit the entire spectrum or parts of it best possible and comment on your results.
  \end{itemize}
%
% https://www.astro.umd.edu/~peel/CPSP118D_101/content/Nature_Pyramids_Article.pdf
% NOTE: The sign difference in alignment due to assumed different stars/use for 5. and 7. has been dropped!!!
\item[5.2] (14 points)
  The table below lists the North-South alignment of Egyptian pyramids (in arc minutes).
  \vspace*{-6ex}
  \begin{center}
  \begin{small}
  \begin{tabular}{lcccccccc}
    \hline
    Pyramid        &1.Meidum       &2.Bent         &3.Red         &4.Khufu       &5.Khafre      &6.Menk.\    &7.Sahure     &8.Nefer.\\
    Align.\ year   &2600 BC        &2583 BC        &2572 BC       &2554 BC       &2522 BC       &2489 BC     &2446 BC      &2433 BC \\
    East Align.    &-$20.6\!\pm\!1.0$  &-$17.3\!\pm\!0.2$  &-$8.7\!\pm\!0.2$  &-$3.4\!\pm\!0.2$  &$6.0\!\pm\!0.2$  &$12.4\!\pm\!1.0$  &$23\!\pm\!10$   &$30\!\pm\!10$\\
    West Align.\   &-$18.1\!\pm\!1.0$  &-$11.8\!\pm\!0.2$  &--                &-$2.8\!\pm\!0.2$  &$6.0\!\pm\!0.2$  &$14.1\!\pm\!1.8$  &--              &--\\
    \hline
  \end{tabular}
  \end{small}
  \end{center}
  \vspace*{-2ex}
  \begin{itemize}
    \item Test to what extend the East (E) and West (W) alignment values agree.
    \item Combine East and West values. Include systematic uncertainties to ensure agreement.
    \item If the alignments were done using circumpolar stars, these drift with Earth's precession.
      Test if the alignment of the pyramids shifts linearly as a function of time.
    \item The astronomically predicted shift as a function of time is 2467 BC + 0.274 arc min./year.
      Does the slope of the linear fit match the predicted value?
    \item What alignment date of Khufu (historically $2554\pm100$ BC) is the astronomical prediction?
  \end{itemize}
\end{description}

\vspace*{-2ex}
\noindent
\hrulefill\\[-4ex]
\begin{center}
\emph{The outcome of a repeated process follows not chance but statistics.}\\
\end{center}

%----------------------------------------------------------------------------

\end{document}

%%% Local Variables: 
%%% mode: latex
%%% TeX-master: t
%%% End: 

